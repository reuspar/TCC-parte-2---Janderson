\label{chapter:Conclusão}

\par
\textcolor{red}{Os resultados obtidos com a árvore de decisão com a utilização do algoritmo J48, se mostrou promissor para ambas as classes da variável dependente Aprovado. Os ramos com maior variação de grau de importância são justamente aqueles que possuem maior variedade de valores para a variável Aprovado. Segundo os resultados do algoritmo J48, as variáveis independentes com mais relevância para a decisão sobre a classe independente, apresentados em ordem decrescente, são: no primeiro nível foi a Deficiência; no segundo nível se Frequentava ou frequenta cursinho pré-vestibular e a Cor de pele; no terceiro nível em diante as variáveis alternaram de posição em função aos valores dos níveis superiores.}

\par
\textcolor{red}{Sobre os candidatos que foram aprovados e que não foram aprovados, com os atributos selecionados como mais relevante pelo WEKA, é observado que, para as ramificações mais percorrida pelos registros das classe da variável Aprovado, indicava os seguintes fatores: para aqueles que não possuíam algum tipo de deficiência, não frequentavam um cursinho pré-vestibular e moram com mais de cinco pessoas, esses três fatores influenciavam para que os candidatos não fossem aprovados, enquanto, para os candidatos que não possuíam alguma deficiência, não frequentava cursinho pré-vestibular, moram com 4 a 5 pessoa e o pai é servidor público, influenciavam para que ele fosse aprovado.}

\par
\textcolor{red}{Para as regras geradas (ramificação pelo qual o registro percorreu),  foi observado que os dados de deficiência, o cursinho frequentado ou não e a quantidade de pessoas que moram com o candidato, são considerados relevantes para o resultado da aprovação dele, isto é, as mesmas características aparecem como determinante para os candidatos que são aprovados e não aprovados.  O mesmo caso ocorre com o trabalho de \citeonline{Simon2017} só que com regras totalmente diferentes, diferente do trabalho de \citeonline{Martinhago2005} que o mesmo não acontece, porém, ele possui algumas regras semelhantes a este trabalho, como é o caso das regras que ele obteve para o curso menos concorrido.}


\par
\textcolor{red}{Sobre a performance do algoritmo J48 para base testada, demonstrou que o modelo não foi considerado bom, pois, ele classificou corretamente apenas 52\% das instancias pertencentes ao conjunto de teste (em relação a classe do resultado final obtido), sendo que se esperava uma taxa de pelo menos 70\% para ele ser considerado um bom modelo, como aconteceu com os trabalhos de \citeonline{Simon2017}  com uma taxa de 77\% de precisão e \citeonline{Martinhago2005} com uma taxa de 89 \% de precisão. Em compensação, a taxa de precisão obtida pelo algoritmo, possui um valor aproximado a taxa de precisão do modelo de validação testado (57\%), uma margem em torno de 10\%.}