\label{chapter:Conclusão}

\par
\textcolor{red}{Os resultados obtidos com a árvore de decisão com a utilização do algoritmo J48, se mostrou promissor para ambas as classes da variável dependente Aprovado. Os ramos com maior variação de grau de importância são justamente aqueles que possuem maior variedade de valores para a variável Aprovado. Segundo os resultados do algoritmo J48, as variáveis independentes com mais relevância para a decisão sobre a classe independente, apresentados em ordem decrescente, são: no primeiro nível foi a Deficiência; no segundo nível se Frequentava ou frequenta cursinho pré-vestibular e a Cor de pele; no terceiro nível em diante as variáveis alternaram de posição em função aos valores dos níveis superiores.}

\par
\textcolor{red}{Sobre os candidatos que foram aprovados e que não foram aprovados, com os atributos selecionados como mais relevante pelo WEKA, é observado que, para as ramificações mais percorrida pelos registros das classe da variável Aprovado, indicava os seguintes fatores: para aqueles que não possuíam algum tipo de deficiência, não frequentavam um cursinho pré-vestibular e moram com mais de cinco pessoas, esses três fatores influenciavam para que os candidatos não fossem aprovados, enquanto, para os candidatos que não possuíam alguma deficiência, não frequentava cursinho pré-vestibular, moram com 4 a 5 pessoa e o pai é servidor público, influenciavam para que ele fosse aprovado.}

\par
\textcolor{red}{}