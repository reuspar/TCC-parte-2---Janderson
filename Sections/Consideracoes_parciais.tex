\label{chapter:consideracoes}

\subsection{Considerações Finais}

\par 
A mineração de dados se tornou uma ferramenta de suporte com papel essencial no gerenciamento da informação dentro das organizações. O manuseamento dos dados e a análise das informações que eram feitas de maneira convencional, se tornou impossível devido a imensa quantidade de dados que é coletado e armazenado em sua base diariamente. Encontrar padrões escondidos e relacionamentos em arquivos que possuem um grande volume de informações de forma manual, deixou de ser uma escolha. As técnicas de mineração passaram a ser utilizadas e começaram a estar presentes no nosso cotidiano.

\par
Com uma grande quantidade de dados que estão sendo armazenados pelas universidades e com recente surgimento de várias tecnologias e técnicas de extração de dados, uma boa parte dessas instituições estão tentando de alguma forma utilizar esses dados que são extraídos  na intenção de compreende-los e utiliza-los para benefício próprio. Como podemos observar nos trabalhos apresentado neste TCC, que através dos dados armazenado de candidatos que vão prestar o vestibular, foi gerado um perfil dos mesmos na intenção de descobrir os motivos que influenciam os seus desempenhos em sua aprovação.

\par
Desta forma, assim como os trabalhos apresentados, que tinham como objetivo de obter os perfis dos candidatos para identificar os fatores que influenciavam na sua aprovação, este trabalho, seguiu com o mesmo intuito tendo como base o software, os algoritmos e os métodos utilizados pelos mesmos. Os resultados obtidos através desses métodos utilizados, se mostraram bastante interessante em relação com a base de dados testada, apresentando perfis com algumas características semelhantes aos trabalhos mencionados. Entretanto, os atributos presentes na base dados em comparação aos trabalhos que foram usados como base, possuem um grau de relevância diferente, assim, como a taxa de precisão da performance dos algoritmos.

\par
O software WEKA durante o uso, demonstrou ser bastante robusto em termos de recursos em relação ao modelo que foi utilizado, o KDD. Sendo que ele apresentou várias utilidades em cada etapa do processo do modelo, desde a etapa do pré-processamento com a modelagem da base de dados até a etapa de pós-processamento, com apresentação dos resultados que foram obtidos através das técnicas e algoritmos implementados a essa API (como gráficos, tabelas, etc). As partes de configuração dos parâmetros dos algoritmos apresentado pelo software, se mostraram bastante intuitivos, facilitando com a manuseamento na parte do processo de mineração dos dados.

\par
Alguns problemas foram encontrados durante o processo de mineração, como o caso de um dos algoritmos utilizados que não conseguiu classificar os registros da base de dados.  Isso aconteceu, pelo fato do atributo principal que indicava quais candidatos foram aprovados, possuírem classes com o seus registros bastante desbalanceados. O problema teve que ser resolvido com o balanceamento dos valores das classes deste atributo, entretanto, teve como consequência uma diminuição da quantidade de registros da base de dados, fazendo com que o algoritmo perdesse um pouco da sua performasse.

\par
Dentre os dois algoritmos testados, o Apriori foi o que demonstrou maior desempenho com a base de dados utilizada, gerando regras bastante precisas, possuindo confiança acima da taxa que foi estabelecida de 70\%. Enquanto o algoritmo J48 obteve uma taxa abaixo do que era desejado, com 53\% de precisão. Para as regras mais confiáveis, os atributos como: o tipo de deficiência do candidato, o estado civil que ele possui, a escola que passou o ensino médio e se ele trabalhava, eram fatores que influenciavam no resultado do desempenho deles na prova, no caso, se eles eram aprovados ou não.  

\par
De acordo com a análise e os parâmetros utilizados neste trabalho, através das regras obtidos pelos algoritmos de classificação e associação, conseguiu ser extraídos informações interessantes através dos perfis gerados pelas regras, tendo como uma taxa de precisão até que boa, em comparação à média entre os dois algoritmos. Dentre essas informações obtidas, analisando os atributos que formavam as regras, foi observado que o atributo que indicava o tipo de deficiência do candidato, teve como maior relevância para as regras obtidas nos dois algoritmos, sendo que em ambos os modelos, foi o atributo que apareceu com mais frequência. Analisando esse atributo mais afundo em relação aos candidatos que foram aprovados, foi observado que uma quantidade bem pequena desses candidatos apareciam nessa classificação, o que demonstra que esse fator pode indicar a falta de acessibilidade no sistema de educação que possuímos hoje.


%\par
%Em resumo, podemos perceber que a mineração de dados vem  cada vez mais sendo utilizada dentro da área educacional como em universidades, pois, com a imensa quantidade de dados que é armazenado, onde muitas vezes é repleta de informações importantes escondidas, se necessita dessas técnicas de mineração para se extrair esses dados com a intenção de usa-las, podendo assim relacionar informações contidas para predeterminar possíveis ações futuras na sua gestão com tutores, alunos e entre outros. Não resta dúvida de que a mineração de dados na área educacional está sendo extremamente promissora e que, apesar dos resultados já alcançados, ainda ela tem muito para o que oferecer.

\subsection{Trabalhos Futuros}

\par
Neste trabalho, foram tratados apenas o desempenho dos candidatos em relação a sua aprovação no vestibular, a partir dos dados socioeconômicos obtidos, contudo, pode ser feita a analise de outros tipos de desempenhos, como é o caso das escolas. Através do histórico escolar de ensino médio dos candidatos que é armazenado pela universidade, pode-se pegar as notas que se encontram nesse histórico, e fazer uma média com elas, para poder verificar o desempenho das escolas em relação ao resultado do desempenho dos alunos no vestibular. 

\par
Outro tipo de analise que pode ser feita, é a chance do candidato de concluir o curso em que ele foi aprovado. Onde, a partir dos perfis que foram gerados, será analisado a probabilidade do candidato em concluir o curso que ele escolheu, através da utilização de um algoritmo treinado com base nos perfis dos acadêmicos que se formaram nos anos anteriores.

\par
Podemos sugerir também, uma análise temporal do desempenho dos candidatos, onde, pegaria a base de dados dos vestibulares anteriores em relação ao ano da base de dados em que está sendo utilizada e identificaria através da mineração, os fatores que mais ocorrem entre esses anos. Analisando assim, o progresso de influência de um determinado fator na aprovação dos candidatos, o período em que mais ocorreu e quais candidatos foram afetados por esses tipos de fatores.
