\label{chapter:projeto}

\par
\textcolor{red}{Os dado...}

\section{Pré-processamento}

\par
\textcolor{red}{Com os dados mencionados anteriormente, foram fornecidos no formato de arquivo .xlsx do Excel com uma tabela contendo três colunas de informações com o ID do candidato, a resposta do questionario e o número do questionario atribuido a resposta, como demonstrado na Figura 19. De 37 questões, somente 33 poderam ser fornecidas, pois, quatro delas tinham como respostas dissertativas (que são armazenadas com as outras também, mas como não possuiam um padrão por serem campos abertos, consequentemente, entrava bastante "lixo") e apenas as 33 tinham como resposta opções já fornecidas pelo próprio sitema que eram armazenadas no banco de dados e que davam de ser disponibilizado.}

\par
\begin{figure}[!htp]
	\begin{center}
    \caption{\label{fig:waveform_fig} Dados fornecidos pela CPV.}
	\includegraphics[scale=0.65]{Figuras/Formato_errado.png}
	\end{center}
    \legend{Fonte: Próprio autor.}
\end{figure}

\par
\textcolor{red}{Para poder começar com a utilização desses dados, foi necessário transformar toda a tabela para um formato apropriado a fim de que os aplicativos do software WEKA os suporte. De início foi feito colunas para cada número do questionário através das funções de filtro e concatenação fornecida pelo Excel, e as linhas abaixo delas seriam as respostas de cada candidato para aquela coluna especifica, como demonstrado na Figura 20. A coluna de ID foi removida já que não era necessário para a mineração, entretanto, ela foi necessária na geração de uma coluna nova rotulada de Aprovado que foi obtida através de duas tabelas contendo informações dos candidatos geral e dos candidatos aprovados.}

\par
\begin{figure}[!htp]
	\begin{center}
    \caption{\label{fig:waveform_fig} Tabela depois de formatada.}
	\includegraphics[scale=0.65]{Figuras/Formato_certo.png}
	\end{center}
    \legend{Fonte: Próprio autor.}
\end{figure}

\par
\textcolor{red}{O motivo dessa formatação demonstrado na Figura 20 é que o WEKA identifica a primeira linha como os atributos e as linhas subsequentes os dados relacionados a esses atributos. Como o software trabalha com formato de arquivo próprio denominado \textit{Atribute-Relation File Format} (arff), foi preciso salvar o arquivo que estava no formato xlsx para o formato csv (que separa colunas por ponto e vírgula), pois, o próprio WEKA converte esse tipo de arquivo para o tipo de arquivo que ele trabalha, o arff.}

\par
\textcolor{red}{Também foi necessário se fazer algumas alterações nas informações de dentro da tabela para que depois da conversão não viesse a ocorrer algum erro nos dados. Alterações essas como a remoção de acentos identificados nas informações da tabela, pois, os arquivos no WEKA segundo \citeonline{Amaral2016}, se encontram no formato ASCII que não suportam caracteres acentuados, outra alteração foi a de transformar números para textos, pois, o tipo de dado em cada coluna só pode ser \textit{numeric nominal}, \textit{string} ou \textit{date}, e por fim, a última alteração foi a da remoção de virgulas encontradas nos dados, pois, para converter um arquivo csv no formato arff, o WEKA identifica as virgulas como as colunas que separam os dados.}

\par
\textcolor{red}{A última alteração que precisou ser feita foi da transformação dos ponto e vírgula para somente virgula, pois, o Excel ao transformar um arquivo em xlsx para csv, ele faz com a separação da coluna seja feita por ponto e vírgula o que acaba ocasionando erros e inconsistência nos dados quando o arquivo é convertido do csv para o arff, pois, o WEKA considera o caractere virgula como a representação das colunas que separam os dados. Após ter feito todos esses processos que foram mencionados anteriormente, a conversão do arquivo foi um sucesso, não tendo sido encontrado nenhum problema e inconsistência nos dados que possam ocasionar erro durante o processo de mineração.}


\subsection{Arquivo ARFF}

\par
\textcolor{red}{Após todo o processo de conversão, internamente, o arquivo arff possui duas seções principais: um cabeçalho e uma área de dados. Segundo \citeonline{Amaral2016}, um cabeçalho deve possuir o nome do conjunto de dados através do atributo relação, no caso, o nome do conjunto deve ser acompanhado pela marca @relation seguido com os atributos que compõem a relação com a marca @attribute. Como podemos ver representado na Figura 21 o arquivo utilizado para a mineração de dados, onde a marca @relation traz o nome do conjunto: Vestibulando de 2017. Se tem no total 34 atributos marcados por @relation, sendo todos os atributos do tipo categórico.} 

\par
\begin{figure}[!htp]
	\begin{center}
    \caption{\label{fig:waveform_fig} Cabeçalho do arquivo arff.}
	\includegraphics[scale=0.57]{Figuras/arquivo_arff.png}
	\end{center}
    \legend{Fonte: Próprio autor.}
\end{figure}

\par
\textcolor{red}{A seção de área de dados tem o seu início definido pela marca @data, onde os dados dessa área são posicionados em linhas, separados por vírgulas, na mesma ordem em que foram colocados os atributos que teve como um total de 8.571 linhas (que é a quantidade de inscritos que responderam o questionário ), como representado na Figura 22.}

\par
\begin{figure}[!htp]
	\begin{center}
    \caption{\label{fig:waveform_fig} Área de Dados do arquivo arff.}
	\includegraphics[scale=0.60]{Figuras/arquivo_arff_2.png}
	\end{center}
    \legend{Fonte: Próprio autor.}
\end{figure}


\subsection{Problemas de classe rara}

\par
\textcolor{red}{É normal em alguns casos de quando se vai minerar algum dado ocorra o problema de classe rara, que segundo \citeonline{Amaral2016}, explica que quando uma instancia de uma classe é predominante do que outra instancia de outra classe, a consequência é de que o modelo aprenderá somente as características da classe dominante, enquanto ela falhará em classificar novas instancias da outra classe, pelo motivo de ela ser uma classe rara. É o que acontece nos dados desse trabalho, especificamente no atributo de Aprovado que tem no total de 8571 instancias, possuindo a classe Sim com 7784 instancias (81\%) e a Classe Nao com 787 instancias (9\%) como demonstrado na Figura 23.}

\par
\begin{figure}[!htp]
	\begin{center}
    \caption{\label{fig:waveform_fig} Representação do atributo Aprovado com as suas duas classes Sim e Nao.}
	\includegraphics[scale=0.90]{Figuras/Atributo_aprovado.png}
	\end{center}
    \legend{Fonte: Próprio autor.}
\end{figure}

\par
\textcolor{red}{Segundo \citeonline{Amaral2016}, a solução mais comum para esse tipo de problema é de utilizar as técnicas de estratificação. No software WEKA, para métodos supervisionado, possui 4 técnicas de estratificação para instancias, no caso, a que foi utilizada para resolver esse tipo de problema foi a técnica SpreadSubsample, pois, segundo o site do software \citeonline{WEKA} esse tipo de filtro gera uma subamostra aleatória de um conjunto de dados, permitindo especificar o \textit{spread} máximo entre a classe mais rara e a classe mais comum, ou seja, balanceado o conjunto de dados dessas duas classe.}

\par
\textcolor{red}{No filtro SpreadSubsample a opção escolhida para a distributionSpread (\textit{spread} máximo distribuição de classe) foi a de 1 que representa a distribuição uniforme como demonstrada na Figura 24, depois de ser aplicada na classe de Aprovado, tem como resultado um conjunto balanceado tanto para a classe mais comum (Sim) como a classe mais rara (Nao) representada na Figura 25.}

\par
\begin{figure}[!htp]
	\begin{center}
    \caption{\label{fig:waveform_fig} Representação do filtro SpreadSubsample.}
	\includegraphics[scale=0.99]{Figuras/SpreadSubsample.png}
	\end{center}
    \legend{Fonte: Próprio autor.}
\end{figure}

\par
\begin{figure}[!htp]
	\begin{center}
    \caption{\label{fig:waveform_fig} Resulato do balanceamento do atributo Aprovado.}
	\includegraphics[scale=0.90]{Figuras/Atributo_aprovado_balanceado.png}
	\end{center}
    \legend{Fonte: Próprio autor.}
\end{figure}

\par
