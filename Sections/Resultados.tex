\label{chapter:Resultados}

\par
\textcolor{red}{Nesta seção serão apresentados os resultados obtidos após as avaliações do modelo, que foi feito com os algoritmos de classificação e associação. Será demonstrado quais variáveis foram classificadas com maior relevância, as ramificações geradas com maior quantidade de registros que percorrem elas, além das regras obtidas para base através do algoritmo de associação.}

\section{Classificação}

\par
\textcolor{red}{Os resultados obtidos com o algoritmo de classificação apresentam um valor mediano de acurácia para o modelo utilizado. A técnica de classificação que foi utilizada, classificou corretamente em torno de 53\% das instâncias que pertence ao conjunto de teste, em relação à classe do resultado que foi obtido. Com base do modelo de árvore de decisão deduzido pelo algoritmo J48, foi possível analisar o grau de interação e a relação das variáveis de acordo com o caminho de interações percorrido pelas respostas do questionário feitas pelos candidatos.}

\par
\textcolor{red}{A árvore obtida apresentou como variável independente principal para a classificação o atributo Deficiencia, que se divide em 5 valores: Nao, Sim visual, Sim outro tipo, Sim fisica e Sim auditiva. As próximas variáveis com maior relevância mudavam de acordo com o valor que era obtido, sendo que, quando a resposta do candidato era de que ele não possuía alguma deficiência ou que ele possuía uma deficiência visual, a variável com maior importância para onde os valores eram direcionados era se ele frequentava ou frequenta cursinho particular, enquanto, para aqueles que possuíam deficiência física ou auditiva eram direcionados para a variável Cor (cor de pele), como apresentada na Figura 41.}

\par
\begin{figure}[!htp]
	\begin{center}
    \caption{\label{fig:waveform_fig} Representação das principais variáveis da árvore de decisão.}
	\includegraphics[scale=0.57]{Figuras/Arvore_gerada_grau2.png}
	\end{center}
    \legend{Fonte: Próprio autor.}
\end{figure}


\par
\textcolor{red}{É possível observar na Figura 42, que para os candidatos que responderam que possuíam outro tipo de deficiência, o valor da resposta era direcionado para um nó folha da classe Nao pertencente a variável dependente Aprovado, isto é, os candidatos que possuíam outro tipo de deficiência que não foram mencionados como opção de resposta da questão, não foram capazes de passar no vestibular (um total de 9 candidatos). A partir do terceiro nível de profundidade, para os candidatos que não possuíam nenhum tipo de deficiência a ramificação da árvore se expandia (tracejado pela linha vermelha), enquanto, diferentemente para os candidatos que possuíam algum tipo de deficiência, eles eram direcionados para os nós folhas da variável Aprovado (tracejado pela linha verde), como apresentado na Figura 40.}

\par
\begin{figure}[!htp]
	\begin{center}
    \caption{\label{fig:waveform_fig} Árvore completa.}
	\includegraphics[scale=0.45]{Figuras/Arvore_completa.png}
	\end{center}
    \legend{Fonte: Próprio autor.}
\end{figure}


\subsection{Resultados Obtidos Para os Candidatos que não Possuiam Nenhum Tipo de Deficiência}


\par
\textcolor{red}{Avaliando a árvore que foi gerada, através da maior quantidade de registros que se encaixam pelo caminho percorrido das ramificações até as folhas, foi obtido os seguintes resultados, começando pela ramificação maior que era a dos candidatos que não possuiam nenhum tipo de deficiencia, a variavel com maior relevancia que foi classificada após o variavel independente principal era se o candidato frequentava um cursinho pré-vestibular. Foi classificado que todos aqueles que frequentaram por mais de um ano algum cursinho, não foram aprovados no vestibular.}

\par
\textcolor{red}{Para aqueles que frequentaram o cursinho por um ano, o atributo correspondente seguinte era a quantidade de pessoas que moravam com ele, sendo que, a maior taxa de candidatos aprovados para essa variavel era quando possuia 3 pessoas morando com ele e para reprovados era quando possuia de 4 a 5 pessoas. Já para os candidatos que ficavam por um semestre fazendo o cursinho, a variavel correspondente seguinte era o meio de transporte que ele utilizava, a maior taxa de aprovados era pra aqueles que usavam o transporte coletivo e para os reprovados eram os outros tipos de transportes não eram mencionados como opção de resposta.}

\par
\textcolor{red}{Para aqueles que não frequentavam algum cursinho pré-vestibular, a variável classificada com maior relevância, seguinte, era a quantidade de pessoas que moravam com o candidato, e dependendo dos valores que eram respondidos para essa variável, os atributos em sequência variavam entre cor de pele, ocupação do pai e meio de transporte utilizado. Foi notado que as maiores taxas de aprovados eram para os candidatos pardos, que moravam de 3 a 5 pessoas e que o pai era servidor público, já para aqueles que não foram aprovados, as maiores taxas eram quando o candidato possuía de quatro ou mais pessoas morando com ele, o pai possuía outro tipo de ocupação e o meio de transporte utilizado era o coletivo.}


\subsection{Resultados Obtidos Para os Candidatos que Possuem Algum Tipo de Deficiência}


\par
\textcolor{red}{Diferente dos valores classificados para os candidatos que não possuíam algum tipo de deficiência, para aqueles que possuíam, o nível de profundidade que a ramificação delas alcançou não foi tão grande quanto a outra. Os resultados obtidos foram, primeiro, para os candidatos que eram deficientes visuais, a taxa de maior aprovação na prova era para aqueles que não frequentaram cursinho pré-vestibular, enquanto para aqueles que frequentaram cursinho por mais de um ano a taxa de reprovação era maior.}

\par
\textcolor{red}{Foi possível observar que tanto para os candidatos com deficiência física quanto para auditiva, a variável que mais influenciava para que eles fossem aprovados ou não era cor de pele. Sendo que para os candidatos com deficiência física, aqueles que eram brancos a chance de ser aprovado era maior, diferente para aqueles que eram pardos tendo sua taxa de reprovação muito alta comparada a outros tons de pele.}

\par
\textcolor{red}{Para os candidatos que possuíam deficiência auditiva, aqueles que tinham um tom de pele pardo obtiveram as maiores taxas de aprovação, ao contrário daqueles que possuíam uma cor de pele branca, a taxa de reprovação era alta. Como já foi mencionado anteriormente na parte de avaliação do trabalho, os candidatos que possuíam qualquer outro tipo de deficiência, foram classificados, que nenhum deles tiveram êxito de serem aprovados no vestibular.}


\section{Associação}

\textcolor{red}{Através dos resultados obtidos com o algoritmo de associação, observou que a medida que o suporte mínimo decrescia, a quantidade de regras que eram geradas aumentava. Os resultados das regras geradas foram divididos em duas partes, para os candidatos aprovados e para os candidatos não aprovados.}

%\textcolor{red}{787 registros aprovados, 7784 registros não aprovados, 8571.}


\subsection{Regras Geradas Para os Candidatos Aprovados}

\par
\textcolor{red}{Analisando as regras geradas em cada teste, na Tabela 5 que apresenta a tabela das regras gerada utilizando 22 atributos com 50\% de suporte, são interpretados da seguinte maneira. Para regra com menor transição de itens, diz que se o candidato possui outro tipo de trabalho não mencionado e a sua residência é própria então ele terá chance de ser aprovado com 100\% de confiança. Já para a maior regra gerada, diz que se o candidato não possui alguma deficiência, seu ensino médio é todo em escola pública e possui outro tipo de trabalho não mencionado, então a chance de ele passar é de 100\% de confiança. Para ambas as regras, acontece com 5\% dos candidatos (394 instancias).}


\par
\begin{table}[!htp]
	\begin{center}
    \caption{\label{fig:waveform_fig} Suporte Mínimo 50\% e Confiança Mínima 70\%.}
	\includegraphics[scale=0.75]{Figuras/Suporte_50_atributos_22.png}
	\end{center}
\end{table}

\par
\textcolor{red}{Para os outros testes com 4 e 8 atributos, com a taxa de 50\% de suporte, nenhuma regra foi gerada para elas. Assim como os resultados do teste de 50\% de suporte, para os testes de 30\% das transições feitas com a quantidade de 4 e 8 atributos, nenhuma regra foi gerada também, já para a base com 22 atributos, foram geradas pelo menos 11 regras no geral, associadas com a variável Aprovados. Sendo que a primeira regra gerada possui 5 itens em sua transição e a última regra gerada chega a possuir 7 itens em sua transição. As regras foram geradas para 236 instancias (3\% dos candidatos), como apresentado na Tabela 6, onde é demonstrado a primeira e a ultima regra gerada.}

\par
\begin{table}[!htp]
	\begin{center}
    \caption{\label{fig:waveform_fig} Suporte Mínimo 30\% e Confiança Mínima 70\%.}
	\includegraphics[scale=0.75]{Figuras/Suporte_30_atributos_22.png}
	\end{center}
\end{table}


\par
\textcolor{red}{Avaliando os testes com 25\% de suporte mínimo, foram geradas regras para as bases com 8 e 22 atributos e para a base com 4 atributos nenhuma regra foi obtida. Analisando a única regra que foi gerada para a base com 8 atributos, diz que 2\% dos candidatos que eram aprovados (197 instancias), eles não trabalhavam, não possuíam algum tipo de deficiência e o meio de transporte utilizado era o coletivo, representado na Tabela 7. Já para a primeira regra gerada para a base com 22 atributos, 2\% dos candidatos que foram aprovados (197 instancias) eram solteiros, sustentados pela família e utilizava o transporte coletivo, como representado na Tabela 8.}

\par
\begin{table}[!htp]
	\begin{center}
    \caption{\label{fig:waveform_fig} Suporte Mínimo 25\% e Confiança Mínima 70\% para a base com 8 atributos.}
	\includegraphics[scale=0.75]{Figuras/Suporte_25_atributos_9.png}
	\end{center}
\end{table}

\par
\begin{table}[!htp]
	\begin{center}
    \caption{\label{fig:waveform_fig} Suporte Mínimo 25\% e Confiança Mínima 70\% para a base com 22 atributos.}
	\includegraphics[scale=0.75]{Figuras/Suporte_25_atributos_22.png}
	\end{center}
\end{table}

\par
\textcolor{red}{Com o suporte mínimo de 15\%, as bases de dados com 4 e 8 atributos obtiveram somente uma regra gerada, enquanto, para a base de dados com 22 atributos, foi obtido 67 regras geradas, sendo que é considerada a maior quantidade de regras que já foi gerada para todos os suportes mínimos testados. Começando com a regra gerada para a base com 4 atributos, como demonstrado na Tabela 9, indica que os candidatos que tem uma mãe com ensino médio completo e possui de 4 a 5 pessoas morando com eles, influenciam para que eles sejam aprovados.}

\par
\begin{table}[!htp]
	\begin{center}
    \caption{\label{fig:waveform_fig} Suporte Mínimo 15\% e Confiança Mínima 70\% para a base com 4 atributos.}
	\includegraphics[scale=0.75]{Figuras/Suporte_15_atributos_4.png}
	\end{center}
\end{table}

\par
\textcolor{red}{Já para a base com 8 atributos, a regra gerada indica que os candidatos que possuem um pai que é servidor público, mora com a família e possui de 4 a 5 pessoas morando com eles, são aprovados no vestibular, como apresentado na Tabela 10.}

\par
\begin{table}[!htp]
	\begin{center}
    \caption{\label{fig:waveform_fig} Suporte Mínimo 15\% e Confiança Mínima 70\% para a base com 8 atributos.}
	\includegraphics[scale=0.75]{Figuras/Suporte_15_atributos_9.png}
	\end{center}
\end{table}

\par
\textcolor{red}{Para a base de dados com 22 atributos com um suporte mínimo de 15\%, conforme é demonstrado da Tabela 11, na primeira regra gerada, a idade de 18 a 21 anos e a mãe como principal responsável, influenciavam para que o candidato fosse aprovado, enquanto para a última regra gerada, os candidatos que eram pardos, que não frequentavam cursinho pré-vestibular, nunca trabalharam e moravam com a família, eles eram aprovados. As regras geradas paras as bases com 4, 9 e 22 atributos, com um suporte de 15\%, influenciavam 1,4\% dos candidatos (118 instancias).}

\par
\begin{table}[!htp]
	\begin{center}
    \caption{\label{fig:waveform_fig} Suporte Mínimo 15\% e Confiança Mínima 70\% para a base com 22 atributos.}
	\includegraphics[scale=0.75]{Figuras/Suporte_15_atributos_22.png}
	\end{center}
\end{table}



\subsection{Regras Geradas Para os Candidatos que não Foram Aprovados}


\par
\textcolor{red}{Para os testes com a base de dados dos candidatos que não foram aprovados, dos 4 suportes mínimos utilizados, apenas 3 geraram regras para as bases de dados, que foram os suportes de 30\%, 25\% e 15\%, sendo que para os suportes testados, geraram regras somente para a base com 22 atributos. Começando com o suporte mínimo de 30\%, como podemos ver na Tabela 12, foi gerado apenas uma regra associada a variável de Aprovados, indicando que que 27\% dos candidatos (2.335 instancias) que não foram aprovados eles eram solteiros, com o tom de pele pardo, não trabalhavam e possuíam residência própria.}


\par
\begin{table}[!htp]
	\begin{center}
    \caption{\label{fig:waveform_fig} Suporte Mínimo 30\% e Confiança Mínima 70\% para a base com 22 atributos.}
	\includegraphics[scale=0.75]{Figuras/Suporte_30_Nao_atributos_22.png}
	\end{center}
\end{table}

\par
\textcolor{red}{Semelhante ao anterior, o suporte mínimo de 25\% gerou uma única regra, somente para a base de 22 atributos, como demonstrado na Tabela 13, a regra gerada indica que 22\% dos candidatos que não foram aprovados (1.946 instancias), eram solteiros, possuem uma renda mensal familiar de 2 a 4 salários mínimos, a residência deles é própria e moram com a família.}

\par
\begin{table}[!htp]
	\begin{center}
    \caption{\label{fig:waveform_fig} Suporte Mínimo 25\% e Confiança Mínima 70\% para a base com 22 atributos.}
	\includegraphics[scale=0.75]{Figuras/Suporte_25_Nao_atributos_22.png}
	\end{center}
\end{table}

\par
\textcolor{red}{Com um suporte mínimo de 15\%, foi obtido regras apenas para a base de dados com 22 atributos, tendo um total de 23 regras geradas. Para a primeira regra, os candidatos que não era deficiente, o ensino médio foi em escola particular e possuía outro tipo de trabalho que não foi mencionado nas opções dadas, eles não eram aprovados. Já para a última regra, para os candidatos com a idade entre 18 a 21 anos, solteiro, não possuía algum tipo de deficiência, o ensino médio foi em escola pública, era sustentado pela família, não trabalhava, possuía uma residência própria e morava com a família, todos esses fatores influenciavam para que eles não fossem aprovados, que era no total de 14\% dos candidatos (1.168 instancias), conforme o que é demonstrado na Tabela 14.}

\par
\begin{table}[!htp]
	\begin{center}
    \caption{\label{fig:waveform_fig} Suporte Mínimo 15\% e Confiança Mínima 70\% para a base com 22 atributos.}
	\includegraphics[scale=0.75]{Figuras/Suporte_15_Nao_atributos_22.png}
	\end{center}
\end{table}


\subsection{Observações nas Taxas de Confiança e Lift Para os Resultados Obtidos}

\par
\textcolor{red}{Nos resultado obtidos com o algoritmo de associação, para todos os casos que foram gerados as regras, a porcentagem de confiança de todas essas regras eram de 100\%, o motivo de todas elas possuírem essa taxa, foi pelo fato de que a base teve que ser dividida em duas partes (candidatos aprovados e não aprovados), para poder retirar a tendência  do algoritmo de gerar regras somente para os candidatos que não foram aprovados, que no caso, eram os que possuíam mais registros. Como o algoritmo foi aplicado em uma base onde a variável Aprovado possuía somente uma classe (Sim ou Nao), as regras geradas associadas para essa variável obtiveram resultados com 100\% de confiança, ou seja, a frequência na qual os atributos aparecem em transações que contenham a classe da variável Aprovado é sempre de 100\%, a para as duas bases. }

\par
\textcolor{red}{Comparando com trabalho de \citeonline{LeandroSilva2014} que serviu como base paras os testes deste trabalho, as regras que foram geradas para a sua base, possuíam taxas que variavam de 70\% a 100\% de confiança. O fator principal que influenciava essas porcentagens, foi porque, a variável principal deles a Nota da prova, continha 4 classes que possuíam quantidades de registros que não variavam muito entre elas, não possuíam uma classe que era mais predominante, ao contrario deste trabalho, que uma das classe possuía 10 vezes mais a quantidade de registro do que a outra classe. Para o caso do parâmetro Lift, que é uma métrica que classifica as melhores regras, o algoritmo classificou para todas as regras geradas, associada a variável Aprovado, o valor 1 de importância, ou seja, todas as regras obtidas possuem o mesmo grau de importância.}

\par
\textcolor{red}{Lembrando que outras regras foram geradas para os mesmos casos testados, onde elas possuíam confiança de 80\% a 90\% e lift abaixo e acima do valor 1, contudo, eles não foram mencionados neste trabalho, pelo fato de não serem associados a variável que é importante para esta pesquisa, no caso, a variável Aprovado.}

\section{Discussão dos resultados}

\par
\textcolor{red}{Os resultados obtidos com a árvore de decisão com a utilização do algoritmo J48, se mostrou promissor para ambas as classes da variável dependente Aprovado. Os ramos com maior variação de grau de importância são justamente aqueles que possuem maior variedade de valores para a variável Aprovado. Segundo os resultados do algoritmo J48, as variáveis independentes com mais relevância para a decisão sobre as classes dependente, apresentados em ordem decrescente, são: no primeiro nível foi a Deficiência; no segundo nível se Frequentava ou frequenta cursinho pré-vestibular e a Cor de pele; no terceiro nível em diante as variáveis alternavam de posição em função aos valores dos níveis superiores.}

\par
\textcolor{red}{Com relação a aprovação dos candidatos, sobre os atributos selecionados como maior relevância pelo WEKA, é observado que, para as ramificações mais percorrida pelos registros das classes da variável Aprovado, indicava os seguintes fatores: para aqueles que não possuíam algum tipo de deficiência, não frequentavam um cursinho pré-vestibular e moravam com mais de cinco pessoas, esses três fatores influenciavam para que os candidatos não fossem aprovados, enquanto, para os candidatos que não possuíam alguma deficiência, não frequentava cursinho pré-vestibular, moravam com 4 a 5 pessoas e também o pai era servidor público, influenciavam para que eles fossem aprovados.}

\par
\textcolor{red}{Para as regras geradas (ramificação pelo qual os registros percorreram),  foi observado que os dados de deficiência, o cursinho frequentado ou não e a quantidade de pessoas que moram com o candidato, são considerados relevantes para o resultado da aprovação dele, isto é, as mesmas características aparecem como determinante para os candidatos que são aprovados e não aprovados.  O mesmo caso ocorre com o trabalho de \citeonline{Simon2017} só que com regras totalmente diferentes, diferente do trabalho de \citeonline{Martinhago2005} que o mesmo não acontece, porém, ele possui algumas regras semelhantes a este trabalho, como é o caso das regras que ele obteve para o curso menos concorrido (se algum frequentou cursinho, quantidades de pessoas que reside com o candidato e a escolaridade da mãe).}


\par
\textcolor{red}{Sobre a performance do algoritmo J48 para base testada, demonstrou que o modelo não foi considerado bom, pois, ele classificou corretamente apenas 52\% das instancias pertencentes ao conjunto de teste (em relação a classe do resultado final obtido), sendo que se esperava uma taxa de pelo menos 70\% para ele ser considerado um bom modelo, como aconteceu com os trabalhos de \citeonline{Simon2017}  com uma taxa de 77\% de precisão e \citeonline{Martinhago2005} com uma taxa de 89 \% de precisão. Em compensação, a taxa de precisão obtida pelo algoritmo, possui um valor aproximado a taxa de precisão do modelo de validação testado (57\%), uma margem em torno de 10\%.}


\par
\textcolor{red}{Para os testes realizados com o algoritmo Apriori, os resultados obtidos, confirmaram o potencial que ele possui, pois, as regras de associação que foram geradas, apresentaram ser bastante consistentes. Com base nos suportes utilizados, foi observado que a medida que o valor deles diminuíam, a quantidade de regras geradas aumentava, o mesmo caso ocorreu com o trabalho de \citeonline{LeandroSilva2014}. Em destaque para a base de dados que possui 22 atributos, onde, a classe de candidatos que foram aprovados, gerou a maior quantidade de regras em comparação com as outras.}

\par
\textcolor{red}{Como já foi mencionado durante a parte de avaliação experimental e analise, os testes foram feitos sobre três tipos de bases diferentes: uma base representando os atributos que foram selecionados por \citeonline{LeandroSilva2014} (4 atributos), outra base representando os atributos que o método de seleção do WEKA selecionou (8 atributos) e a base com todos os atributos que permaneceram, após a remoção de atributos desnecessário para a pesquisa (22 atributos). Para o suporte mínimo com 15\%, mostrou bastante eficiência em gerar regras para as 3 bases, enquanto para os suportes com 25\%, 30\% e 50\% não obtiveram o mesmo êxito para as mesmas bases.}

\par
\textcolor{red}{Observando os testes realizados para a base de dados com 4 atributos, de todos os suporte testados, apenas uma regra foi gerada, sendo ela, para o suporte com 15\%. O algoritmo não mostrou muita eficiência para os atributos utilizados por \citeonline{LeandroSilva2014} em relação  aos registros pertencentes a base de dados utilizada neste projeto, entretanto, a regra que foi gerada é semelhante a mesma regra obtida por eles com o mesmo suporte utilizado (15\%), onde, a escolaridade da mãe e a quantidade de pessoas que residem com o candidato, influenciavam para que o candidato possuísse um desempenho regular na prova do ENEM.}

\par
\textcolor{red}{Já para os testes da base de dados com 8 atributos, foram gerados apenas uma regra para o suporte cada um dos suportes, com 25\% e 15\%, sendo elas relacionadas ao candidatos que foram aprovados. As regras para esses dois tipos de suporte não apresentaram características semelhantes, sendo, que mesmo elas possuindo bases iguais, os itens que formam a regra é totalmente diferente para cada suporte. Comparado com os resultados obtidos com a mesma base, utilizado nos testes com árvore de decisão, as regras geradas pelo Apriori possuem diferenças com as regras geradas pelo J48, tendo como semelhante somente o atributo deficiência com o grau de relevância.}

\par
\textcolor{red}{Para os testes da base com 22 atributos, foi onde o algoritmo teve mais eficiência, gerando regras tanto para os candidatos que foram aprovados quanto os que não foram. Para todos os suportes testados que tiveram como resultado os candidatos que foram aprovados (com exceção do suporte de 50\% que não obteve nenhuma regra), os itens que mais se destacaram nas regras obtidas foram o tipo de deficiência, o estado civil do candidato, o tipo de escola que o mesmo frequentou no ensino médio, se ele trabalhava e o tipo de ocupação dele, esses 5 fatores influenciavam na sua aprovação. Enquanto para os candidatos não aprovados, os itens que mais se destacaram nas regras geradas foi o tipo de deficiência do candidato, o tipo de residência que ele possui, o estado civil dele e se o candidato trabalhava.}

\par
\textcolor{red}{Analisando os fatores obtidos para a classe Não do atributo Aprovados através das respostas dos candidatos (referente a base de dados com 22 atributos), mesmo que eles não trabalhassem, não eram comprometidos e possuíam uma residência própria, ainda assim, uma boa parte deles com esses fatores, tinha uma alta taxa de reprovação, o que pode acabar indicando o desinteresse deles em de se preparar para a prova do vestibular, ou, o desinteresse deles em relação aos estudos nas escolas.}

\par
\textcolor{red}{Em comparação com os dois algoritmos testados, foi observado que, mesmo que o modelo de árvore de decisão possuísse regras que fazem o total sentido, dentro do contexto dos resultados do candidatos, as regras que foram consideradas mais confiáveis para a obtenção dos perfis dos que foram aprovados e não aprovados, foi o modelo de associação do algoritmo Apriori. Comparando a execução dos dois modelos, o Apriori se mostrou mais eficiente para a base dados utilizada deste projeto do que o J48, possuindo uma taxa de confiança de 100\% para as regras obtidas, enquanto para as regras da árvore de decisão, obteve 53\% de precisão para no que foi gerado (registros classificados corretamente para essas regras).}

\textcolor{red}{Contundo, mesmo com uma diferença grande de precisão entre os dois algoritmos, o atributo que mais se destacou nas regras obtidas, foi o tipo de deficiência que o candidato possui (aparecendo para ambas as classes do atributo Aprovado, sim e não). Analisando mais a fundo os dados da base dados em relação a esse atributo, percebeu que apenas 3\% (27 registros) dos candidatos que foram aprovados, eram deficientes, representando um valor muito abaixo se comparado com o restante que foi aprovado. O que pode acabar indicando a falta de acessibilidade em ralação as escolas ou forma que é aplicada a prova para esse tipo de grupo. Infelizmente não se obteve informações sobre os dados dos resultados dos anos anteriores, para fazer comparação, com objetivo de saber se é uma taxa boa ou não comparado aos outros anos.}